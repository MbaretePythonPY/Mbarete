\documentclass[10pt]{article}
\usepackage{babel}
\usepackage{amsmath}
\title{\LaTeX}
\date{}
% Este es un comentario, no será mostrado en el documento final.
\begin{document}
\LaTeX{} es un programa para preparar documentos con  el sistema de
tipograf\'ias\footnote{%nota al pie de página Seg\'un Wikipedia, la 
tipograf\'ia es el arte y t\'ecnica del manejo y selecci\'on de tipos, 
originalmente de plomo, para crear trabajos de impresi\'on } %fin nota al pie de página
\TeX{}. \LaTeX{} fue desarrollado originalmente por Leslie Lamport en 
1984 y se convirti\'o en el m\'etodo dominante para la  manipulaci\'on 
de \TeX. La versi\'on utilizada para generar  este documento es \LaTeXe.
\newline
\begin{align*}
f(x) &= (x+1)^2 \\
&= x^2 + x + x + 1 \\
&= x^2 + 2x + 1 \\
\textbf{código } \textbf{símbolo}\\
\text{Alpha } \mathrm{A}, \text{ Beta } \mathrm{B}, \text{ Gamma } \Gamma, \text{ Delta } \Delta\\
E &= mc^2                              \\
m &= \frac{m_0}{\sqrt{1-\frac{v^2}{c^2}}} \\
\text{encuadrado} = \begin{array}{|c|} \hline 0=a_{11} + a_{12}\\ \hline \end{array}\\
\text{exponentes} = x^{a+b}=x^ax^b \\
\text{raiz} = x_i=\sqrt[n]{\frac{a_i}{b_i}}
\end{align*}
\begin{align*}
\text{sumatoria } = \sum_{0\le i\le m, 0<j<n}P(i, j) 
\end{align*}
\begin{align*}
\text{oint } = \oint F(x)dx 
\end{align*}
\begin{align*}
\text{iint } = \iint \Phi(x, y)dxdy 
\end{align*}
\end{document}
\begin{document}
\begin{align*}
\text{factorial } = {n \choose r} = \frac{n!}{r! (n - r)!} 
\end{align*}
\begin{align*}
\text{superindices } = x'+x'' = \dot x + \ddot x 
\end{align*}
\begin{align*}
\text{vectores } = \vec{\mathbf{v}} = a\hat x + b\hat y 
\end{align*}
\begin{align*}
\text{segmentos } = \overline{AB} \subset \bar{C} 
\end{align*}
\begin{align*}
\text{matriz_cuadrilatero } = \begin{matrix}A\xrightarrow{\;\;\;f\;\;\;}B \pi\downarrow{\;\;\;\;\;}\;\;\;\uparrow{} \phi C\xrightarrow{\;\;\;g\;\;\;}D\end{matrix} 
\end{align*}
\begin{align*}
\text{limite } = \lim{n \rightarrow \infty} \frac{n \cdot l}{2 \cdot r} = \pi 
\end{align*}
\begin{align*}
\text{matriz} = \begin{pmatrix}\alpha & \cdots & \beta^{*}\\\vdots & \ddots & \vdots\\ \gamma^{*} & \cdots & \delta \end{pmatrix}
\end{align*}
\begin{align*}
\text{integral } =\int_{\vert x-x_0 \vert < X_0}\Phi(x)
\end{align*}
\begin{align*}
\text{integral Limite } = \int\limits_{\vert x-x_0 \vert < X_0}\Phi(x) 
\end{align*}
\begin{align*}
\text{funcion_partida } = f(x)=\begin{cases} 0 & \text{ si } x>0 \\ x^2 & \text{ si no } \end{cases} \\
\text{signo parentesis } = \big( \Big( \bigg( \Bigg( \quad \\
\text{signo llave } = \big\} \Big\} \bigg\} \Bigg\}\quad \\
\text{signo lineV } = \big\| \Big\| \bigg\| \Bigg\| \quad \\
\text{signo downArrow } &= \big\Downarrow \Big\Downarrow \bigg\Downarrow \Bigg\Downarrow \\
\text{fracciones} = <math{{sust:ns:0}}>\tfrac{\cfrac{1}{2}\dfrac{3}{4}\frac{3}{4}\tfrac{7}{8}}{2}</math> \\
\end{align*}
\end{document}